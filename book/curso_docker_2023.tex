% Options for packages loaded elsewhere
\PassOptionsToPackage{unicode}{hyperref}
\PassOptionsToPackage{hyphens}{url}
\PassOptionsToPackage{dvipsnames,svgnames,x11names}{xcolor}
%
\documentclass[
  a4paper,
  DIV=11,
  numbers=noendperiod,
  onepage,
  openany]{scrreprt}

\usepackage{amsmath,amssymb}
\usepackage{iftex}
\ifPDFTeX
  \usepackage[T1]{fontenc}
  \usepackage[utf8]{inputenc}
  \usepackage{textcomp} % provide euro and other symbols
\else % if luatex or xetex
  \usepackage{unicode-math}
  \defaultfontfeatures{Scale=MatchLowercase}
  \defaultfontfeatures[\rmfamily]{Ligatures=TeX,Scale=1}
\fi
\usepackage{lmodern}
\ifPDFTeX\else  
    % xetex/luatex font selection
\fi
% Use upquote if available, for straight quotes in verbatim environments
\IfFileExists{upquote.sty}{\usepackage{upquote}}{}
\IfFileExists{microtype.sty}{% use microtype if available
  \usepackage[]{microtype}
  \UseMicrotypeSet[protrusion]{basicmath} % disable protrusion for tt fonts
}{}
\makeatletter
\@ifundefined{KOMAClassName}{% if non-KOMA class
  \IfFileExists{parskip.sty}{%
    \usepackage{parskip}
  }{% else
    \setlength{\parindent}{0pt}
    \setlength{\parskip}{6pt plus 2pt minus 1pt}}
}{% if KOMA class
  \KOMAoptions{parskip=half}}
\makeatother
\usepackage{xcolor}
\usepackage[lmargin=30mm,rmargin=30mm,tmargin=35mm,bmargin=30mm]{geometry}
\setlength{\emergencystretch}{3em} % prevent overfull lines
\setcounter{secnumdepth}{5}
% Make \paragraph and \subparagraph free-standing
\ifx\paragraph\undefined\else
  \let\oldparagraph\paragraph
  \renewcommand{\paragraph}[1]{\oldparagraph{#1}\mbox{}}
\fi
\ifx\subparagraph\undefined\else
  \let\oldsubparagraph\subparagraph
  \renewcommand{\subparagraph}[1]{\oldsubparagraph{#1}\mbox{}}
\fi

\usepackage{color}
\usepackage{fancyvrb}
\newcommand{\VerbBar}{|}
\newcommand{\VERB}{\Verb[commandchars=\\\{\}]}
\DefineVerbatimEnvironment{Highlighting}{Verbatim}{commandchars=\\\{\}}
% Add ',fontsize=\small' for more characters per line
\usepackage{framed}
\definecolor{shadecolor}{RGB}{241,243,245}
\newenvironment{Shaded}{\begin{snugshade}}{\end{snugshade}}
\newcommand{\AlertTok}[1]{\textcolor[rgb]{0.68,0.00,0.00}{#1}}
\newcommand{\AnnotationTok}[1]{\textcolor[rgb]{0.37,0.37,0.37}{#1}}
\newcommand{\AttributeTok}[1]{\textcolor[rgb]{0.40,0.45,0.13}{#1}}
\newcommand{\BaseNTok}[1]{\textcolor[rgb]{0.68,0.00,0.00}{#1}}
\newcommand{\BuiltInTok}[1]{\textcolor[rgb]{0.00,0.23,0.31}{#1}}
\newcommand{\CharTok}[1]{\textcolor[rgb]{0.13,0.47,0.30}{#1}}
\newcommand{\CommentTok}[1]{\textcolor[rgb]{0.37,0.37,0.37}{#1}}
\newcommand{\CommentVarTok}[1]{\textcolor[rgb]{0.37,0.37,0.37}{\textit{#1}}}
\newcommand{\ConstantTok}[1]{\textcolor[rgb]{0.56,0.35,0.01}{#1}}
\newcommand{\ControlFlowTok}[1]{\textcolor[rgb]{0.00,0.23,0.31}{#1}}
\newcommand{\DataTypeTok}[1]{\textcolor[rgb]{0.68,0.00,0.00}{#1}}
\newcommand{\DecValTok}[1]{\textcolor[rgb]{0.68,0.00,0.00}{#1}}
\newcommand{\DocumentationTok}[1]{\textcolor[rgb]{0.37,0.37,0.37}{\textit{#1}}}
\newcommand{\ErrorTok}[1]{\textcolor[rgb]{0.68,0.00,0.00}{#1}}
\newcommand{\ExtensionTok}[1]{\textcolor[rgb]{0.00,0.23,0.31}{#1}}
\newcommand{\FloatTok}[1]{\textcolor[rgb]{0.68,0.00,0.00}{#1}}
\newcommand{\FunctionTok}[1]{\textcolor[rgb]{0.28,0.35,0.67}{#1}}
\newcommand{\ImportTok}[1]{\textcolor[rgb]{0.00,0.46,0.62}{#1}}
\newcommand{\InformationTok}[1]{\textcolor[rgb]{0.37,0.37,0.37}{#1}}
\newcommand{\KeywordTok}[1]{\textcolor[rgb]{0.00,0.23,0.31}{#1}}
\newcommand{\NormalTok}[1]{\textcolor[rgb]{0.00,0.23,0.31}{#1}}
\newcommand{\OperatorTok}[1]{\textcolor[rgb]{0.37,0.37,0.37}{#1}}
\newcommand{\OtherTok}[1]{\textcolor[rgb]{0.00,0.23,0.31}{#1}}
\newcommand{\PreprocessorTok}[1]{\textcolor[rgb]{0.68,0.00,0.00}{#1}}
\newcommand{\RegionMarkerTok}[1]{\textcolor[rgb]{0.00,0.23,0.31}{#1}}
\newcommand{\SpecialCharTok}[1]{\textcolor[rgb]{0.37,0.37,0.37}{#1}}
\newcommand{\SpecialStringTok}[1]{\textcolor[rgb]{0.13,0.47,0.30}{#1}}
\newcommand{\StringTok}[1]{\textcolor[rgb]{0.13,0.47,0.30}{#1}}
\newcommand{\VariableTok}[1]{\textcolor[rgb]{0.07,0.07,0.07}{#1}}
\newcommand{\VerbatimStringTok}[1]{\textcolor[rgb]{0.13,0.47,0.30}{#1}}
\newcommand{\WarningTok}[1]{\textcolor[rgb]{0.37,0.37,0.37}{\textit{#1}}}

\providecommand{\tightlist}{%
  \setlength{\itemsep}{0pt}\setlength{\parskip}{0pt}}\usepackage{longtable,booktabs,array}
\usepackage{calc} % for calculating minipage widths
% Correct order of tables after \paragraph or \subparagraph
\usepackage{etoolbox}
\makeatletter
\patchcmd\longtable{\par}{\if@noskipsec\mbox{}\fi\par}{}{}
\makeatother
% Allow footnotes in longtable head/foot
\IfFileExists{footnotehyper.sty}{\usepackage{footnotehyper}}{\usepackage{footnote}}
\makesavenoteenv{longtable}
\usepackage{graphicx}
\makeatletter
\def\maxwidth{\ifdim\Gin@nat@width>\linewidth\linewidth\else\Gin@nat@width\fi}
\def\maxheight{\ifdim\Gin@nat@height>\textheight\textheight\else\Gin@nat@height\fi}
\makeatother
% Scale images if necessary, so that they will not overflow the page
% margins by default, and it is still possible to overwrite the defaults
% using explicit options in \includegraphics[width, height, ...]{}
\setkeys{Gin}{width=\maxwidth,height=\maxheight,keepaspectratio}
% Set default figure placement to htbp
\makeatletter
\def\fps@figure{htbp}
\makeatother

\KOMAoption{captions}{tableheading}
\makeatletter
\@ifpackageloaded{tcolorbox}{}{\usepackage[skins,breakable]{tcolorbox}}
\@ifpackageloaded{fontawesome5}{}{\usepackage{fontawesome5}}
\definecolor{quarto-callout-color}{HTML}{909090}
\definecolor{quarto-callout-note-color}{HTML}{0758E5}
\definecolor{quarto-callout-important-color}{HTML}{CC1914}
\definecolor{quarto-callout-warning-color}{HTML}{EB9113}
\definecolor{quarto-callout-tip-color}{HTML}{00A047}
\definecolor{quarto-callout-caution-color}{HTML}{FC5300}
\definecolor{quarto-callout-color-frame}{HTML}{acacac}
\definecolor{quarto-callout-note-color-frame}{HTML}{4582ec}
\definecolor{quarto-callout-important-color-frame}{HTML}{d9534f}
\definecolor{quarto-callout-warning-color-frame}{HTML}{f0ad4e}
\definecolor{quarto-callout-tip-color-frame}{HTML}{02b875}
\definecolor{quarto-callout-caution-color-frame}{HTML}{fd7e14}
\makeatother
\makeatletter
\makeatother
\makeatletter
\@ifpackageloaded{bookmark}{}{\usepackage{bookmark}}
\makeatother
\makeatletter
\@ifpackageloaded{caption}{}{\usepackage{caption}}
\AtBeginDocument{%
\ifdefined\contentsname
  \renewcommand*\contentsname{Table of contents}
\else
  \newcommand\contentsname{Table of contents}
\fi
\ifdefined\listfigurename
  \renewcommand*\listfigurename{List of Figures}
\else
  \newcommand\listfigurename{List of Figures}
\fi
\ifdefined\listtablename
  \renewcommand*\listtablename{List of Tables}
\else
  \newcommand\listtablename{List of Tables}
\fi
\ifdefined\figurename
  \renewcommand*\figurename{Figure}
\else
  \newcommand\figurename{Figure}
\fi
\ifdefined\tablename
  \renewcommand*\tablename{Table}
\else
  \newcommand\tablename{Table}
\fi
}
\@ifpackageloaded{float}{}{\usepackage{float}}
\floatstyle{ruled}
\@ifundefined{c@chapter}{\newfloat{codelisting}{h}{lop}}{\newfloat{codelisting}{h}{lop}[chapter]}
\floatname{codelisting}{Listing}
\newcommand*\listoflistings{\listof{codelisting}{List of Listings}}
\makeatother
\makeatletter
\@ifpackageloaded{caption}{}{\usepackage{caption}}
\@ifpackageloaded{subcaption}{}{\usepackage{subcaption}}
\makeatother
\makeatletter
\@ifpackageloaded{tcolorbox}{}{\usepackage[skins,breakable]{tcolorbox}}
\makeatother
\makeatletter
\@ifundefined{shadecolor}{\definecolor{shadecolor}{rgb}{.97, .97, .97}}
\makeatother
\makeatletter
\makeatother
\makeatletter
\makeatother
\ifLuaTeX
  \usepackage{selnolig}  % disable illegal ligatures
\fi
\IfFileExists{bookmark.sty}{\usepackage{bookmark}}{\usepackage{hyperref}}
\IfFileExists{xurl.sty}{\usepackage{xurl}}{} % add URL line breaks if available
\urlstyle{same} % disable monospaced font for URLs
\hypersetup{
  pdftitle={Docker 2023},
  pdfauthor={Diego Saavedra},
  colorlinks=true,
  linkcolor={blue},
  filecolor={Maroon},
  citecolor={Blue},
  urlcolor={Blue},
  pdfcreator={LaTeX via pandoc}}

\title{Docker 2023}
\author{Diego Saavedra}
\date{Dec 4, 2023}

\begin{document}
\maketitle
\ifdefined\Shaded\renewenvironment{Shaded}{\begin{tcolorbox}[enhanced, boxrule=0pt, breakable, interior hidden, frame hidden, sharp corners, borderline west={3pt}{0pt}{shadecolor}]}{\end{tcolorbox}}\fi

\renewcommand*\contentsname{Table of contents}
{
\hypersetup{linkcolor=}
\setcounter{tocdepth}{2}
\tableofcontents
}
\bookmarksetup{startatroot}

\hypertarget{curso-de-docker}{%
\chapter{Curso de Docker}\label{curso-de-docker}}

¡Bienvenidos al Curso de Docker!

Este curso te guiará a través de un viaje desde los fundamentos hasta el
dominio de Docker, la plataforma de contenedores líder en la industria.

\hypertarget{quuxe9-es-este-curso}{%
\section{¿Qué es este Curso?}\label{quuxe9-es-este-curso}}

Este curso exhaustivo te llevará desde los conceptos básicos de Docker
hasta la implementación práctica de aplicaciones y servicios en
contenedores. A través de una combinación de teoría y ejercicios
prácticos, explorarás cada aspecto diseñado para fortalecer tus
habilidades en el uso de Docker. Desde la instalación inicial hasta la
orquestación de contenedores con Docker Compose, este curso te
proporcionará las herramientas y conocimientos necesarios para
desarrollar, desplegar y escalar aplicaciones con eficiencia.

\hypertarget{a-quiuxe9n-estuxe1-dirigido}{%
\section{¿A quién está dirigido?}\label{a-quiuxe9n-estuxe1-dirigido}}

Este curso está diseñado tanto para aquellos que están dando sus
primeros pasos en Docker como para aquellos que desean profundizar en
sus conocimientos. Ya seas un estudiante, un profesional en busca de
nuevas habilidades o alguien apasionado por la tecnología de
contenedores, este curso te brindará la base necesaria para trabajar de
manera efectiva con Docker en cualquier entorno.

\hypertarget{cuxf3mo-contribuir}{%
\section{¿Cómo contribuir?}\label{cuxf3mo-contribuir}}

Valoramos tu participación en este curso. Si encuentras errores, deseas
sugerir mejoras o agregar contenido adicional, ¡nos encantaría recibir
tus contribuciones! Puedes contribuir a través de nuestra plataforma en
línea, donde puedes compartir tus comentarios y sugerencias. Juntos,
podemos mejorar continuamente este recurso educativo para beneficiar a
la comunidad de usuarios de Docker.

Este curso ha sido creado con el objetivo de proporcionar acceso
gratuito y universal al conocimiento de Docker. Estará disponible en
línea para que cualquiera, sin importar su ubicación o circunstancias,
pueda acceder y aprender a su propio ritmo.

¡Esperamos que disfrutes este emocionante viaje de aprendizaje y
descubrimiento en el mundo de Docker y los contenedores!

\part{Curso Docker}

\hypertarget{comandos-buxe1sicos-y-atajos}{%
\chapter{1. Comandos Básicos y
Atajos}\label{comandos-buxe1sicos-y-atajos}}

\hypertarget{conceptos}{%
\section{Conceptos:}\label{conceptos}}

\textbf{Docker:} Es una plataforma para desarrollar, enviar y ejecutar
aplicaciones en contenedores. Un contenedor es una instancia ejecutable
de una imagen.

\textbf{Contenedor:} Es una instancia de una imagen que se ejecuta de
manera aislada. Los contenedores son ligeros y portátiles, ya que
incluyen todo lo necesario para ejecutar una aplicación, incluidas las
bibliotecas y las dependencias.

\hypertarget{ejemplos}{%
\section{Ejemplos:}\label{ejemplos}}

Descargar una imagen:

\begin{Shaded}
\begin{Highlighting}[]
\ExtensionTok{docker}\NormalTok{ pull docker/getting{-}started}
\end{Highlighting}
\end{Shaded}

Este comando descarga la imagen \textbf{getting-started} desde el
registro público de Docker.

Correr un contenedor en el puerto 80:

\begin{Shaded}
\begin{Highlighting}[]
\ExtensionTok{docker}\NormalTok{ run }\AttributeTok{{-}d} \AttributeTok{{-}p}\NormalTok{ 80:80 docker/getting{-}started}
\end{Highlighting}
\end{Shaded}

Este comando ejecuta un contenedor desenlazado en segundo plano (-d) y
mapea el puerto 80 de la máquina host al puerto 80 del contenedor (-p
80:80).

\hypertarget{pruxe1ctica}{%
\section{Práctica:}\label{pruxe1ctica}}

Descarga la imagen ``nginx'':

\begin{Shaded}
\begin{Highlighting}[]
\ExtensionTok{docker}\NormalTok{ pull nginx}
\end{Highlighting}
\end{Shaded}

Descarga la imagen de Nginx desde el registro público.

Crea y ejecuta un contenedor en el puerto 8080:

\begin{Shaded}
\begin{Highlighting}[]
\ExtensionTok{docker}\NormalTok{ run }\AttributeTok{{-}d} \AttributeTok{{-}p}\NormalTok{ 8080:80 nginx}
\end{Highlighting}
\end{Shaded}

\begin{itemize}
\tightlist
\item
  Crea y ejecuta un contenedor de Nginx en el puerto 8080.
\item
  Detén y elimina el contenedor creado:
\item
  Utiliza los comandos para detener y eliminar un contenedor.
\end{itemize}

Resolución de la Actividad Práctica

Descarga la imagen de \textbf{Nginx}.

\begin{Shaded}
\begin{Highlighting}[]
\ExtensionTok{docker}\NormalTok{ pull nginx}
\end{Highlighting}
\end{Shaded}

Crea y ejecuta un contenedor en el puerto 8080:

\begin{Shaded}
\begin{Highlighting}[]
\ExtensionTok{docker}\NormalTok{ run }\AttributeTok{{-}d} \AttributeTok{{-}p}\NormalTok{ 8080:80 nginx}
\end{Highlighting}
\end{Shaded}

Detén y elimina el contenedor creado:

\begin{Shaded}
\begin{Highlighting}[]
\ExtensionTok{docker}\NormalTok{ stop }\VariableTok{$(}\ExtensionTok{docker}\NormalTok{ ps }\AttributeTok{{-}q}\VariableTok{)}
\ExtensionTok{docker}\NormalTok{ rm }\VariableTok{$(}\ExtensionTok{docker}\NormalTok{ ps }\AttributeTok{{-}a} \AttributeTok{{-}q}\VariableTok{)}
\end{Highlighting}
\end{Shaded}

\begin{tcolorbox}[enhanced jigsaw, opacityback=0, titlerule=0mm, bottomtitle=1mm, arc=.35mm, toptitle=1mm, breakable, colframe=quarto-callout-tip-color-frame, left=2mm, leftrule=.75mm, coltitle=black, rightrule=.15mm, toprule=.15mm, colbacktitle=quarto-callout-tip-color!10!white, colback=white, bottomrule=.15mm, title=\textcolor{quarto-callout-tip-color}{\faLightbulb}\hspace{0.5em}{Tip}, opacitybacktitle=0.6]

Combinar banderas mejora la eficiencia en la ejecución de comandos.

\end{tcolorbox}

\hypertarget{quuxe9-aprendimos}{%
\section{¿Qué Aprendimos?}\label{quuxe9-aprendimos}}

\begin{itemize}
\tightlist
\item
  Aprendimos a descargar imágenes, correr contenedores y gestionarlos
  básicamente.
\item
  Entendimos la importancia de las banderas en los comandos Docker.
\end{itemize}

\hypertarget{actividad-pruxe1ctica}{%
\chapter{Actividad Práctica}\label{actividad-pruxe1ctica}}

\hypertarget{objetivo}{%
\section{Objetivo:}\label{objetivo}}

Familiarizarse con los comandos básicos de Docker y atajos para
gestionar contenedores de manera eficiente. Instrucciones:

\begin{itemize}
\tightlist
\item
  Utilizando el comando docker run, inicia un contenedor de la imagen
  ``nginx'' en segundo plano, mapeando el puerto 8080 del host al puerto
  80 del contenedor.
\item
  Detén y elimina el contenedor recién creado utilizando comandos
  Docker.
\item
  Crea un nuevo contenedor con la imagen ``alpine'' y ejecuta un
  terminal interactivo dentro de él.
\item
  Desde el contenedor alpine, instala el paquete curl utilizando el
  gestor de paquetes apk.
\item
  Crea una imagen llamada ``alpine-curl'' a partir de este contenedor
  modificado.
\end{itemize}

\hypertarget{entregables}{%
\section{Entregables:}\label{entregables}}

\begin{itemize}
\tightlist
\item
  Documento explicando los comandos utilizados.
\item
  Imagen Docker ``alpine-curl'' disponible localmente.
\end{itemize}

\hypertarget{rubrica-de-evaluaciuxf3n}{%
\section{Rubrica de Evaluación:}\label{rubrica-de-evaluaciuxf3n}}

\begin{itemize}
\tightlist
\item
  Correcta ejecución de comandos: 6 puntos
\item
  Clara documentación: 4 puntos
\item
  Imagen ``alpine-curl'' creada correctamente: 10 puntos
\end{itemize}

Resolución de la Actividad Práctica

\begin{itemize}
\tightlist
\item
  Iniciar un contenedor Nginx:
\end{itemize}

\begin{Shaded}
\begin{Highlighting}[]
\ExtensionTok{docker}\NormalTok{ run }\AttributeTok{{-}d} \AttributeTok{{-}p}\NormalTok{ 8080:80 }\AttributeTok{{-}{-}name}\NormalTok{ my{-}nginx nginx}
\end{Highlighting}
\end{Shaded}

Detener y eliminar el contenedor Nginx:

\begin{Shaded}
\begin{Highlighting}[]
\ExtensionTok{docker}\NormalTok{ stop my{-}nginx}
\ExtensionTok{docker}\NormalTok{ rm my{-}nginx}
\end{Highlighting}
\end{Shaded}

Crear un contenedor Alpine interactivo:

\begin{Shaded}
\begin{Highlighting}[]
\ExtensionTok{docker}\NormalTok{ run }\AttributeTok{{-}it} \AttributeTok{{-}{-}name}\NormalTok{ my{-}alpine alpine /bin/sh}
\end{Highlighting}
\end{Shaded}

Instalar el paquete curl desde el contenedor Alpine:

\begin{Shaded}
\begin{Highlighting}[]
\ExtensionTok{apk}\NormalTok{ add }\AttributeTok{{-}{-}no{-}cache}\NormalTok{ curl}
\end{Highlighting}
\end{Shaded}

Crear una nueva imagen ``alpine-curl'':

\begin{Shaded}
\begin{Highlighting}[]
\ExtensionTok{docker}\NormalTok{ commit my{-}alpine alpine{-}curl}
\end{Highlighting}
\end{Shaded}

\hypertarget{trabajando-con-voluxfamenes}{%
\chapter{2: Trabajando con
Volúmenes}\label{trabajando-con-voluxfamenes}}

\hypertarget{conceptos-1}{%
\section{Conceptos:}\label{conceptos-1}}

\textbf{Volumes:} Son mecanismos que permiten persistir datos más allá
del ciclo de vida de un contenedor.

\hypertarget{ejemplos-1}{%
\section{Ejemplos:}\label{ejemplos-1}}

Crear un nuevo volumen:

\begin{Shaded}
\begin{Highlighting}[]
\ExtensionTok{docker}\NormalTok{ volume create my\_volume}
\end{Highlighting}
\end{Shaded}

Este comando crea un volumen llamado ``my\_volume''.

Usar un volumen al correr un contenedor:

\begin{Shaded}
\begin{Highlighting}[]
\ExtensionTok{docker}\NormalTok{ run }\AttributeTok{{-}v}\NormalTok{ my\_volume:/app }\AttributeTok{{-}d}\NormalTok{ node:18{-}alpine}
\end{Highlighting}
\end{Shaded}

Este comando ejecuta un contenedor de Node.js y vincula el volumen
``my\_volume'' al directorio ``/app'' dentro del contenedor.

\hypertarget{pruxe1ctica-1}{%
\section{Práctica:}\label{pruxe1ctica-1}}

\begin{itemize}
\tightlist
\item
  Crea un volumen llamado \textbf{data\_volume}.
\item
  Crea un volumen llamado \textbf{data\_volume}.
\item
  Ejecuta un contenedor de MySQL, utilizando el volumen creado.
\end{itemize}

Resolución de la Actividad Práctica

Crea un volumen:

\begin{Shaded}
\begin{Highlighting}[]
\ExtensionTok{docker}\NormalTok{ volume create data\_volume}
\end{Highlighting}
\end{Shaded}

Ejecuta el contenedor MySQL con el volumen creado:

\begin{Shaded}
\begin{Highlighting}[]
\ExtensionTok{docker}\NormalTok{ run }\AttributeTok{{-}e}\NormalTok{ MYSQL\_ROOT\_PASSWORD=root }\AttributeTok{{-}e}\NormalTok{ MYSQL\_DATABASE=mydb }\AttributeTok{{-}v}\NormalTok{ data\_volume:/var/lib/mysql }\AttributeTok{{-}{-}network{-}alias}\NormalTok{ mysql }\AttributeTok{{-}d}\NormalTok{ mysql:latest}
\end{Highlighting}
\end{Shaded}

\begin{tcolorbox}[enhanced jigsaw, opacityback=0, titlerule=0mm, bottomtitle=1mm, arc=.35mm, toptitle=1mm, breakable, colframe=quarto-callout-tip-color-frame, left=2mm, leftrule=.75mm, coltitle=black, rightrule=.15mm, toprule=.15mm, colbacktitle=quarto-callout-tip-color!10!white, colback=white, bottomrule=.15mm, title=\textcolor{quarto-callout-tip-color}{\faLightbulb}\hspace{0.5em}{Tip}, opacitybacktitle=0.6]

\textbf{Recuerda la regla de oro:} Si dos contenedores están en la misma
red, podrán comunicarse.

\end{tcolorbox}

\hypertarget{quuxe9-aprendimos-1}{%
\section{¿Qué Aprendimos?}\label{quuxe9-aprendimos-1}}

\begin{itemize}
\tightlist
\item
  Ahora entendemos cómo trabajar con volúmenes para mantener la
  persistencia de datos en Docker.
\item
  Aprendimos sobre Named Volumes y cómo vincularlos a contenedores.
\end{itemize}

\hypertarget{actividad-pruxe1ctica-1}{%
\chapter{Actividad Práctica}\label{actividad-pruxe1ctica-1}}

\hypertarget{objetivo-1}{%
\section{Objetivo:}\label{objetivo-1}}

Practicar el uso de volúmenes en Docker para persistir datos entre
contenedores.

\hypertarget{instrucciones}{%
\section{Instrucciones:}\label{instrucciones}}

\begin{itemize}
\tightlist
\item
  Crea un nuevo volumen llamado ``mydata''.
\item
  Inicia un contenedor de la imagen ``nginx'' y vincula el volumen
  ``mydata'' al directorio ``/usr/share/nginx/html'' dentro del
  contenedor.
\item
  Crea un archivo HTML dentro del volumen ``mydata'' con el mensaje
  ``Hola, este es un archivo HTML persistente''.
\item
  Inicia otro contenedor de la imagen ``nginx'' y vincula el mismo
  volumen ``mydata'' al directorio ``/usr/share/nginx/html'' dentro de
  este segundo contenedor.
\item
  Verifica que ambos contenedores comparten el mismo archivo HTML creado
  en el paso 3.
\end{itemize}

\hypertarget{entregables-1}{%
\section{Entregables:}\label{entregables-1}}

\begin{itemize}
\tightlist
\item
  Documento explicando los comandos utilizados.
\item
  Capturas de pantalla que demuestren la persistencia de datos entre
  contenedores.
\end{itemize}

\hypertarget{rubrica-de-evaluaciuxf3n-1}{%
\section{Rubrica de Evaluación:}\label{rubrica-de-evaluaciuxf3n-1}}

\begin{itemize}
\tightlist
\item
  Correcta creación y vinculación de volúmenes: 6 puntos
\item
  Creación y persistencia de archivos en el volumen: 8 puntos
\item
  Verificación exitosa de la persistencia entre contenedores: 6 puntos
\end{itemize}

Resolución de la Actividad Práctica

Crear un nuevo volumen:

\begin{Shaded}
\begin{Highlighting}[]
\ExtensionTok{docker}\NormalTok{ volume create mydata}
\end{Highlighting}
\end{Shaded}

Iniciar el primer contenedor Nginx con el volumen:

\begin{Shaded}
\begin{Highlighting}[]
\ExtensionTok{docker}\NormalTok{ run }\AttributeTok{{-}d} \AttributeTok{{-}p}\NormalTok{ 8080:80 }\AttributeTok{{-}{-}name}\NormalTok{ nginx{-}1 }\AttributeTok{{-}v}\NormalTok{ mydata:/usr/share/nginx/html nginx}
\end{Highlighting}
\end{Shaded}

Crear un archivo HTML dentro del volumen:

\begin{Shaded}
\begin{Highlighting}[]
\ExtensionTok{docker}\NormalTok{ exec }\AttributeTok{{-}it}\NormalTok{ nginx{-}1 sh }\AttributeTok{{-}c} \StringTok{"echo \textquotesingle{}Hola, este es un archivo HTML persistente\textquotesingle{} \textgreater{} /usr/share/nginx/html/index.html"}
\end{Highlighting}
\end{Shaded}

Iniciar el segundo contenedor Nginx con el mismo volumen:

\begin{Shaded}
\begin{Highlighting}[]
\ExtensionTok{docker}\NormalTok{ run }\AttributeTok{{-}d} \AttributeTok{{-}p}\NormalTok{ 8081:80 }\AttributeTok{{-}{-}name}\NormalTok{ nginx{-}2 }\AttributeTok{{-}v}\NormalTok{ mydata:/usr/share/nginx/html nginx}
\end{Highlighting}
\end{Shaded}

Verificar la persistencia del archivo HTML:

\begin{itemize}
\tightlist
\item
  Acceder a \url{http://localhost:8080} en el navegador.
\item
  Acceder a \url{http://localhost:8081} en el navegador.
\end{itemize}

\hypertarget{dockerfile-y-docker-compose}{%
\chapter{3: Dockerfile y Docker
Compose}\label{dockerfile-y-docker-compose}}

\hypertarget{conceptos-2}{%
\section{Conceptos:}\label{conceptos-2}}

\textbf{Dockerfile:} Es un archivo de texto que contiene instrucciones
para construir una imagen Docker. Es como un plano para la construcción
de imágenes.

\textbf{Docker Compose:} Es una herramienta que permite definir y
compartir aplicaciones multi-contenedor. Con un solo archivo
(docker-compose.yml), puedes configurar y ejecutar tus servicios.

\hypertarget{ejemplos-2}{%
\section{Ejemplos:}\label{ejemplos-2}}

\begin{itemize}
\tightlist
\item
  Crear un Dockerfile para una aplicación Node.js:
\end{itemize}

\begin{Shaded}
\begin{Highlighting}[]
\KeywordTok{FROM}\NormalTok{ node:14}
\KeywordTok{WORKDIR}\NormalTok{ /app}
\KeywordTok{COPY}\NormalTok{ . .}
\KeywordTok{CMD}\NormalTok{ [}\StringTok{"npm"}\NormalTok{, }\StringTok{"start"}\NormalTok{]}
\end{Highlighting}
\end{Shaded}

Este Dockerfile configura una imagen de Node.js, establece el directorio
de trabajo, copia los archivos locales al contenedor y define el comando
para ejecutar la aplicación.

\begin{itemize}
\tightlist
\item
  Configurar Docker Compose para una aplicación Node.js:
\end{itemize}

\begin{Shaded}
\begin{Highlighting}[]
\AttributeTok{    }\FunctionTok{version}\KeywordTok{:}\AttributeTok{ }\StringTok{\textquotesingle{}3\textquotesingle{}}
\AttributeTok{    }\FunctionTok{services}\KeywordTok{:}
\AttributeTok{      }\FunctionTok{myapp}\KeywordTok{:}
\AttributeTok{        }\FunctionTok{build}\KeywordTok{:}
\AttributeTok{          }\FunctionTok{context}\KeywordTok{:}\AttributeTok{ .}
\AttributeTok{          }\FunctionTok{dockerfile}\KeywordTok{:}\AttributeTok{ Dockerfile.node}
\AttributeTok{        }\FunctionTok{image}\KeywordTok{:}\AttributeTok{ my{-}node{-}app}
\end{Highlighting}
\end{Shaded}

Este archivo docker-compose.yml define un servicio llamado ``myapp'' que
construirá una imagen usando el Dockerfile ``Dockerfile.node'' y le
asignará el nombre de ``my-node-app''.

\hypertarget{pruxe1ctica-2}{%
\section{Práctica:}\label{pruxe1ctica-2}}

\begin{itemize}
\item
  Crea un Dockerfile para una aplicación Python simple:
\item
  Configura un archivo docker-compose.yml para ejecutar la aplicación:

  Resolución de la Actividad Práctica
\item
  Ejemplo de Dockerfile (nombre: Dockerfile.python):
\end{itemize}

\begin{Shaded}
\begin{Highlighting}[]

\KeywordTok{FROM}\NormalTok{ python:3.9}
\KeywordTok{WORKDIR}\NormalTok{ /app}
\KeywordTok{COPY}\NormalTok{ . .}
\KeywordTok{CMD}\NormalTok{ [}\StringTok{"python"}\NormalTok{, }\StringTok{"app.py"}\NormalTok{]}
\end{Highlighting}
\end{Shaded}

Ejemplo de docker-compose.yml:

\begin{Shaded}
\begin{Highlighting}[]
\FunctionTok{version}\KeywordTok{:}\AttributeTok{ }\StringTok{\textquotesingle{}3\textquotesingle{}}
\FunctionTok{services}\KeywordTok{:}
\AttributeTok{  }\FunctionTok{myapp}\KeywordTok{:}
\AttributeTok{    }\FunctionTok{build}\KeywordTok{:}
\AttributeTok{      }\FunctionTok{context}\KeywordTok{:}\AttributeTok{ .}
\AttributeTok{      }\FunctionTok{dockerfile}\KeywordTok{:}\AttributeTok{ Dockerfile.python}
\AttributeTok{    }\FunctionTok{image}\KeywordTok{:}\AttributeTok{ my{-}python{-}app}
\end{Highlighting}
\end{Shaded}

\begin{tcolorbox}[enhanced jigsaw, opacityback=0, titlerule=0mm, bottomtitle=1mm, arc=.35mm, toptitle=1mm, breakable, colframe=quarto-callout-tip-color-frame, left=2mm, leftrule=.75mm, coltitle=black, rightrule=.15mm, toprule=.15mm, colbacktitle=quarto-callout-tip-color!10!white, colback=white, bottomrule=.15mm, title=\textcolor{quarto-callout-tip-color}{\faLightbulb}\hspace{0.5em}{Tip}, opacitybacktitle=0.6]

Cuando trabajas con Docker Compose, es útil conocer el comando
docker-compose up con la opción -d para ejecutar los contenedores en
segundo plano. Esto permite liberar la terminal para otras operaciones
mientras tus servicios continúan ejecutándose en el fondo.

\begin{Shaded}
\begin{Highlighting}[]
\ExtensionTok{docker{-}compose}\NormalTok{ up }\AttributeTok{{-}d}
\end{Highlighting}
\end{Shaded}

Este comando es especialmente útil en entornos de desarrollo donde
deseas ejecutar múltiples servicios, pero aún así necesitas utilizar tu
terminal para otras tareas. Además, puedes detener los servicios en
segundo plano con:

\begin{Shaded}
\begin{Highlighting}[]
\ExtensionTok{docker{-}compose}\NormalTok{ down}
\end{Highlighting}
\end{Shaded}

Esto ayudará a liberar los recursos utilizados por los contenedores sin
afectar tu entorno de desarrollo principal.

\end{tcolorbox}

\hypertarget{quuxe9-aprendimos-2}{%
\section{¿Qué Aprendimos?}\label{quuxe9-aprendimos-2}}

\begin{itemize}
\tightlist
\item
  Hemos adquirido habilidades para crear imágenes personalizadas y
  gestionar aplicaciones multi-contenedor con Docker Compose.
\item
  Ahora comprendemos la importancia de organizar nuestras aplicaciones
  en contenedores y cómo Docker Compose simplifica la orquestación.
\end{itemize}

\hypertarget{actividad-pruxe1ctica-2}{%
\chapter{Actividad Práctica}\label{actividad-pruxe1ctica-2}}

\hypertarget{objetivo-2}{%
\section{Objetivo:}\label{objetivo-2}}

Practicar la creación de imágenes personalizadas utilizando Dockerfile y
la orquestación de servicios con Docker Compose.

\hypertarget{instrucciones-1}{%
\section{Instrucciones:}\label{instrucciones-1}}

\begin{itemize}
\tightlist
\item
  Crea un Dockerfile para una aplicación Python simple que imprima
  ``Hola, Docker'' al ejecutarse.
\item
  Construye la imagen a partir del Dockerfile.
\item
  Utilizando Docker Compose, define un servicio que utilice la imagen
  creada y exponga el puerto 5000.
\item
  Inicia el servicio con Docker Compose.
\item
  Accede a la aplicación en \url{http://localhost:5000} y verifica que
  imprime ``Hola, Docker''.
\end{itemize}

\hypertarget{entregables-2}{%
\section{Entregables:}\label{entregables-2}}

\begin{itemize}
\tightlist
\item
  Dockerfile para la aplicación Python.
\item
  Archivo Docker Compose.
\item
  Documento explicando los comandos utilizados.
\item
  Capturas de pantalla que demuestren el acceso a la aplicación.
\end{itemize}

\hypertarget{rubrica-de-evaluaciuxf3n-2}{%
\section{Rubrica de Evaluación:}\label{rubrica-de-evaluaciuxf3n-2}}

\begin{itemize}
\tightlist
\item
  Correcta creación del Dockerfile: 6 puntos
\item
  Imagen construida correctamente: 4 puntos
\item
  Configuración adecuada en Docker Compose: 6 puntos
\item
  Acceso exitoso a la aplicación: 4 puntos
\end{itemize}

Resolución de la Actividad Práctica

Dockerfile para la aplicación Python:

\begin{Shaded}
\begin{Highlighting}[]
\KeywordTok{FROM}\NormalTok{ python:3.9}
\KeywordTok{CMD}\NormalTok{ [}\StringTok{"python"}\NormalTok{, }\StringTok{"{-}c"}\NormalTok{, }\StringTok{"print(\textquotesingle{}Hola, Docker\textquotesingle{})"}\NormalTok{]}
\end{Highlighting}
\end{Shaded}

Construir la imagen:

\begin{Shaded}
\begin{Highlighting}[]
\ExtensionTok{docker}\NormalTok{ build }\AttributeTok{{-}t}\NormalTok{ my{-}python{-}app .}
\end{Highlighting}
\end{Shaded}

Archivo Docker Compose (docker-compose.yml):

\begin{Shaded}
\begin{Highlighting}[]
\FunctionTok{version}\KeywordTok{:}\AttributeTok{ }\StringTok{\textquotesingle{}3\textquotesingle{}}
\FunctionTok{services}\KeywordTok{:}
\AttributeTok{  }\FunctionTok{myapp}\KeywordTok{:}
\AttributeTok{    }\FunctionTok{image}\KeywordTok{:}\AttributeTok{ my{-}python{-}app}
\AttributeTok{    }\FunctionTok{ports}\KeywordTok{:}
\AttributeTok{      }\KeywordTok{{-}}\AttributeTok{ }\StringTok{"5000:5000"}
\end{Highlighting}
\end{Shaded}

Iniciar el servicio con Docker Compose:

\begin{Shaded}
\begin{Highlighting}[]
\ExtensionTok{docker{-}compose}\NormalTok{ up }\AttributeTok{{-}d}
\end{Highlighting}
\end{Shaded}

Verificar el acceso a la aplicación:

Acceder a \url{http://localhost:5000} en el navegador.

\hypertarget{buenas-pruxe1cticas-y-seguridad}{%
\chapter{4: Buenas Prácticas y
Seguridad}\label{buenas-pruxe1cticas-y-seguridad}}

\hypertarget{conceptos-3}{%
\section{Conceptos:}\label{conceptos-3}}

\textbf{Escaneo de imagen:} Después de construir una imagen, es buena
práctica realizar un escaneo en busca de vulnerabilidades.

\textbf{Capas de la imagen:} Cada imagen de Docker se construye en
capas, permitiendo un nivel de abstracción independiente.

\textbf{Multi-Stage builds:} Permiten separar dependencias necesarias
para construir la aplicación de las necesarias para ejecutarla en
producción, reduciendo el tamaño de la imagen final.

\hypertarget{ejemplos-3}{%
\section{Ejemplos:}\label{ejemplos-3}}

Escaneo de imagen con Snyk:

\begin{Shaded}
\begin{Highlighting}[]
\ExtensionTok{snyk}\NormalTok{ container test }\OperatorTok{\textless{}}\NormalTok{IMAGE\_NAME:TAG}\OperatorTok{\textgreater{}}
\end{Highlighting}
\end{Shaded}

Utilizando la herramienta Snyk, podemos escanear una imagen en busca de
vulnerabilidades.

Uso de Multi-Stage builds:

\begin{Shaded}
\begin{Highlighting}[]
\KeywordTok{FROM}\NormalTok{ node:14 }\KeywordTok{AS}\NormalTok{ builder}
\KeywordTok{WORKDIR}\NormalTok{ /app}
  \KeywordTok{COPY}\NormalTok{ . .}
  \KeywordTok{RUN} \ExtensionTok{npm}\NormalTok{ install}
  \KeywordTok{RUN} \ExtensionTok{npm}\NormalTok{ run build}

  \KeywordTok{FROM}\NormalTok{ nginx:alpine}
  \KeywordTok{COPY} \OperatorTok{{-}{-}from=builder}\NormalTok{ /app/dist /usr/share/nginx/html}
\end{Highlighting}
\end{Shaded}

Este Dockerfile utiliza Multi-Stage builds para primero construir una
aplicación Node.js y luego copiar solo los artefactos necesarios en una
imagen más ligera de Nginx.

\hypertarget{pruxe1ctica-3}{%
\section{Práctica:}\label{pruxe1ctica-3}}

\begin{itemize}
\tightlist
\item
  Realiza un escaneo de vulnerabilidades en una imagen de tu elección:
\item
  Utiliza Snyk para escanear una imagen de Docker.
\item
  Implementa un Multi-Stage build en un Dockerfile:
\end{itemize}

Resolución de la Actividad Práctica

Escaneo de imagen con Snyk:

\begin{Shaded}
\begin{Highlighting}[]
\ExtensionTok{snyk}\NormalTok{ container test my{-}image:my{-}tag}
\end{Highlighting}
\end{Shaded}

Ejemplo de Multi-Stage Dockerfile:

\begin{Shaded}
\begin{Highlighting}[]

\KeywordTok{FROM}\NormalTok{ node:14 }\KeywordTok{AS}\NormalTok{ builder}
\KeywordTok{WORKDIR}\NormalTok{ /app}
  \KeywordTok{COPY}\NormalTok{ . .}
  \KeywordTok{RUN} \ExtensionTok{npm}\NormalTok{ install}
  \KeywordTok{RUN} \ExtensionTok{npm}\NormalTok{ run build}

  \KeywordTok{FROM}\NormalTok{ nginx:alpine}
  \KeywordTok{COPY} \OperatorTok{{-}{-}from=builder}\NormalTok{ /app/dist /usr/share/nginx/html}
\end{Highlighting}
\end{Shaded}

\begin{tcolorbox}[enhanced jigsaw, opacityback=0, titlerule=0mm, bottomtitle=1mm, arc=.35mm, toptitle=1mm, breakable, colframe=quarto-callout-tip-color-frame, left=2mm, leftrule=.75mm, coltitle=black, rightrule=.15mm, toprule=.15mm, colbacktitle=quarto-callout-tip-color!10!white, colback=white, bottomrule=.15mm, title=\textcolor{quarto-callout-tip-color}{\faLightbulb}\hspace{0.5em}{Tip}, opacitybacktitle=0.6]

Recuerda crear contenedores efímeros y desacoplar aplicaciones para
mejorar el rendimiento y la eficiencia

\end{tcolorbox}

\hypertarget{quuxe9-aprendimos-3}{%
\section{¿Qué Aprendimos?}\label{quuxe9-aprendimos-3}}

\begin{itemize}
\tightlist
\item
  Entendemos la importancia de escanear imágenes en busca de
  vulnerabilidades.
\item
  Ahora sabemos cómo utilizar Multi-Stage builds para optimizar el
  tamaño de las imágenes.
\end{itemize}

\hypertarget{actividad-pruxe1ctica-3}{%
\chapter{Actividad Práctica}\label{actividad-pruxe1ctica-3}}

\hypertarget{objetivo-3}{%
\section{Objetivo:}\label{objetivo-3}}

Aplicar buenas prácticas y medidas de seguridad al trabajar con Docker.

\hypertarget{instrucciones-2}{%
\section{Instrucciones:}\label{instrucciones-2}}

\begin{itemize}
\tightlist
\item
  Escanea la imagen ``alpine:latest'' en busca de vulnerabilidades
  utilizando la herramienta Snyk.
\item
  Implementa una política de etiquetado para las imágenes Docker que
  siga las mejores prácticas.
\item
  Utiliza la herramienta ``docker-compose lint'' para verificar la
  validez del archivo Docker Compose.
\item
  Implementa una red de contenedores y asegúrate de que solo los
  contenedores necesarios tengan acceso a ella.
\item
  Crea un archivo ``.dockerignore'' para excluir archivos y directorios
  innecesarios en la construcción de imágenes Docker.
\end{itemize}

\hypertarget{entregables-3}{%
\section{Entregables:}\label{entregables-3}}

\begin{itemize}
\tightlist
\item
  Capturas de pantalla del escaneo de vulnerabilidades.
\item
  Política de etiquetado para imágenes Docker.
\item
  Resultado de la verificación del archivo Docker Compose.
\item
  Documento explicando la implementación de la red de contenedores.
\item
  Archivo ``.dockerignore''.
\end{itemize}

\hypertarget{rubrica-de-evaluaciuxf3n-3}{%
\section{Rubrica de Evaluación:}\label{rubrica-de-evaluaciuxf3n-3}}

\begin{itemize}
\tightlist
\item
  Escaneo de vulnerabilidades realizado con éxito: 4 puntos
\item
  Correcta implementación de la política de etiquetado: 4 puntos
\item
  Validación exitosa del archivo Docker Compose: 4 puntos
\item
  Implementación adecuada de la red de contenedores: 4 puntos
\item
  Correcta configuración del archivo ``.dockerignore'': 4 puntos
\end{itemize}

Resolución de la Actividad Práctica

Escaneo de vulnerabilidades:

\begin{Shaded}
\begin{Highlighting}[]
\ExtensionTok{snyk}\NormalTok{ container test alpine:latest}
\end{Highlighting}
\end{Shaded}

Política de etiquetado (ejemplo):

Se etiquetarán las imágenes con el formato ``versión-año-mes-día''
(ejemplo: 1.0-20230115).

Verificación del archivo Docker Compose:

\begin{Shaded}
\begin{Highlighting}[]
\ExtensionTok{docker{-}compose}\NormalTok{ config}
\end{Highlighting}
\end{Shaded}

Implementación de una red de contenedores:

\begin{Shaded}
\begin{Highlighting}[]
\ExtensionTok{docker}\NormalTok{ network create my{-}network}
\end{Highlighting}
\end{Shaded}

Archivo ``.dockerignore'' (ejemplo):

\begin{Shaded}
\begin{Highlighting}[]
\ExtensionTok{node\_modules}
\ExtensionTok{.git}
\ExtensionTok{.env}
\end{Highlighting}
\end{Shaded}

\hypertarget{devcontainers}{%
\chapter{5. DevContainers}\label{devcontainers}}

\hypertarget{conceptos-4}{%
\section{Conceptos:}\label{conceptos-4}}

\textbf{DevContainers:} DevContainers es una característica de Visual
Studio Code que permite definir, configurar y compartir fácilmente
entornos de desarrollo con contenedores de Docker. Con DevContainers,
puedes especificar las herramientas, extensiones y configuraciones
necesarias para tu proyecto, garantizando que todos los miembros del
equipo utilicen el mismo entorno de desarrollo, independientemente de su
sistema operativo.

\textbf{Stack MEAN:} El stack MEAN (MongoDB, Express.js, Angular,
Node.js) es un conjunto de tecnologías ampliamente utilizado para el
desarrollo web moderno. MongoDB es una base de datos NoSQL, Express.js
es un marco web para Node.js, Angular es un framework para construir
aplicaciones web de una sola página (SPA), y Node.js es un entorno de
ejecución para JavaScript en el lado del servidor.

\hypertarget{ejemplos-4}{%
\section{Ejemplos:}\label{ejemplos-4}}

Configuración de DevContainers para el stack MEAN:

Creación del archivo .devcontainer/devcontainer.json:

\begin{Shaded}
\begin{Highlighting}[]
\FunctionTok{\{}
  \DataTypeTok{"name"}\FunctionTok{:} \StringTok{"MEAN Stack"}\FunctionTok{,}
  \DataTypeTok{"dockerComposeFile"}\FunctionTok{:} \StringTok{"docker{-}compose.yml"}\FunctionTok{,}
  \DataTypeTok{"service"}\FunctionTok{:} \StringTok{"app"}\FunctionTok{,}
  \DataTypeTok{"workspaceFolder"}\FunctionTok{:} \StringTok{"/workspace"}\FunctionTok{,}
  \DataTypeTok{"extensions"}\FunctionTok{:} \OtherTok{[}
    \StringTok{"ms{-}vscode.node{-}debug2"}\OtherTok{,}
    \StringTok{"dbaeumer.vscode{-}eslint"}
  \OtherTok{]}\FunctionTok{,}
  \DataTypeTok{"postCreateCommand"}\FunctionTok{:} \StringTok{"npm install"}
\FunctionTok{\}}
\end{Highlighting}
\end{Shaded}

Creación del archivo docker-compose.yml:

\begin{Shaded}
\begin{Highlighting}[]
\FunctionTok{version}\KeywordTok{:}\AttributeTok{ }\StringTok{\textquotesingle{}3\textquotesingle{}}
\FunctionTok{services}\KeywordTok{:}
\AttributeTok{  }\FunctionTok{app}\KeywordTok{:}
\AttributeTok{    }\FunctionTok{image}\KeywordTok{:}\AttributeTok{ node:14}
\AttributeTok{    }\FunctionTok{command}\KeywordTok{:}\AttributeTok{ /bin/sh {-}c "while sleep 1000; do :; done"}
\AttributeTok{    }\FunctionTok{volumes}\KeywordTok{:}
\AttributeTok{      }\KeywordTok{{-}}\AttributeTok{ ..:/workspace}
\end{Highlighting}
\end{Shaded}

Estructura del proyecto:

\begin{Shaded}
\begin{Highlighting}[]
\NormalTok{.}
\NormalTok{├── .devcontainer}
\NormalTok{│   └── devcontainer.json}
\NormalTok{├── docker{-}compose.yml}
\NormalTok{└── src}
\NormalTok{    └── app.js}
\end{Highlighting}
\end{Shaded}

Archivo app.js (Ejemplo de aplicación Node.js):

\begin{Shaded}
\begin{Highlighting}[]
\KeywordTok{const}\NormalTok{ express }\OperatorTok{=} \PreprocessorTok{require}\NormalTok{(}\StringTok{\textquotesingle{}express\textquotesingle{}}\NormalTok{)}\OperatorTok{;}
\KeywordTok{const}\NormalTok{ app }\OperatorTok{=} \FunctionTok{express}\NormalTok{()}\OperatorTok{;}
\KeywordTok{const}\NormalTok{ port }\OperatorTok{=} \DecValTok{3000}\OperatorTok{;}

\NormalTok{app}\OperatorTok{.}\FunctionTok{get}\NormalTok{(}\StringTok{\textquotesingle{}/\textquotesingle{}}\OperatorTok{,}\NormalTok{ (req}\OperatorTok{,}\NormalTok{ res) }\KeywordTok{=\textgreater{}}\NormalTok{ res}\OperatorTok{.}\FunctionTok{send}\NormalTok{(}\StringTok{\textquotesingle{}Hola desde Express.js en un contenedor Docker!\textquotesingle{}}\NormalTok{))}\OperatorTok{;}

\NormalTok{app}\OperatorTok{.}\FunctionTok{listen}\NormalTok{(port}\OperatorTok{,}\NormalTok{ () }\KeywordTok{=\textgreater{}} \BuiltInTok{console}\OperatorTok{.}\FunctionTok{log}\NormalTok{(}\VerbatimStringTok{\textasciigrave{}Aplicación escuchando en http://localhost:}\SpecialCharTok{$\{}\NormalTok{port}\SpecialCharTok{\}}\VerbatimStringTok{\textasciigrave{}}\NormalTok{))}\OperatorTok{;}
\end{Highlighting}
\end{Shaded}

\hypertarget{actividad-pruxe1ctica-4}{%
\section{Actividad Práctica:}\label{actividad-pruxe1ctica-4}}

Configurar un entorno de desarrollo MEAN con DevContainers en Visual
Studio Code.

\begin{itemize}
\tightlist
\item
  Clona un repositorio base que contenga la estructura mencionada y el
  código de ejemplo.
\item
  Abre el proyecto en Visual Studio Code.
\item
  Visualiza y comprende los archivos .devcontainer/devcontainer.json y
  docker-compose.yml.
\item
  Inicia el entorno de desarrollo con DevContainers.
\item
  Accede a la aplicación en tu navegador utilizando el puerto
  especificado en el archivo app.js.
\item
  Realiza modificaciones en el código de la aplicación y observa cómo se
  reflejan en tiempo real dentro del contenedor.
\end{itemize}

Resolución de la Actividad Práctica

Clonar el repositorio base:

\begin{Shaded}
\begin{Highlighting}[]
\FunctionTok{git}\NormalTok{ clone https://ejemplo{-}repositorio{-}mean.git}
\end{Highlighting}
\end{Shaded}

Abrir el proyecto en Visual Studio Code:

\begin{Shaded}
\begin{Highlighting}[]
\ExtensionTok{code}\NormalTok{ ejemplo{-}repositorio{-}mean}
\end{Highlighting}
\end{Shaded}

Iniciar el entorno de desarrollo con DevContainers:

Visual Studio Code detectará automáticamente la configuración de
DevContainers y te preguntará si deseas reabrir el proyecto en un
contenedor.

Acceder a la aplicación en el navegador:

Visita \url{http://localhost:3000} en tu navegador.

Realizar modificaciones en el código:

Abre el archivo src/app.js y realiza cambios en el mensaje de respuesta.

Observa cómo los cambios se reflejan en tiempo real dentro del
contenedor.

\begin{tcolorbox}[enhanced jigsaw, opacityback=0, titlerule=0mm, bottomtitle=1mm, arc=.35mm, toptitle=1mm, breakable, colframe=quarto-callout-tip-color-frame, left=2mm, leftrule=.75mm, coltitle=black, rightrule=.15mm, toprule=.15mm, colbacktitle=quarto-callout-tip-color!10!white, colback=white, bottomrule=.15mm, title=\textcolor{quarto-callout-tip-color}{\faLightbulb}\hspace{0.5em}{Tip}, opacitybacktitle=0.6]

Si experimentas problemas con la integración de DevContainers, asegúrate
de tener Docker instalado y la extensión ``Remote - Containers''
habilitada en Visual Studio Code.

Además, verifica que tu sistema cumple con los requisitos para
DevContainers.

\end{tcolorbox}

\hypertarget{quuxe9-aprendimos-4}{%
\section{¿Qué Aprendimos?}\label{quuxe9-aprendimos-4}}

\begin{itemize}
\tightlist
\item
  Aprendimos a configurar DevContainers en Visual Studio Code para un
  entorno de desarrollo MEAN.
\item
  Comprendimos cómo utilizar Docker Compose junto con DevContainers para
  definir la infraestructura del entorno de desarrollo.
\item
  Exploramos la posibilidad de realizar cambios en el código de la
  aplicación de manera eficiente gracias a la integración de
  DevContainers con Visual Studio Code.
\end{itemize}

\hypertarget{actividad-pruxe1ctica-5}{%
\chapter{Actividad Práctica}\label{actividad-pruxe1ctica-5}}

\hypertarget{objetivo-4}{%
\section{Objetivo:}\label{objetivo-4}}

Aplicar buenas prácticas y medidas de seguridad al trabajar con Docker.

\hypertarget{instrucciones-3}{%
\section{Instrucciones:}\label{instrucciones-3}}

\begin{itemize}
\tightlist
\item
  Escanea la imagen ``alpine:latest'' en busca de vulnerabilidades
  utilizando la herramienta Snyk.
\item
  Implementa una política de etiquetado para las imágenes Docker que
  siga las mejores prácticas.
\item
  Utiliza la herramienta ``docker-compose lint'' para verificar la
  validez del archivo Docker Compose.
\item
  Implementa una red de contenedores y asegúrate de que solo los
  contenedores necesarios tengan acceso a ella.
\item
  Crea un archivo ``.dockerignore'' para excluir archivos y directorios
  innecesarios en la construcción de imágenes Docker.
\end{itemize}

\hypertarget{entregables-4}{%
\section{Entregables:}\label{entregables-4}}

\begin{itemize}
\tightlist
\item
  Capturas de pantalla del escaneo de vulnerabilidades.
\item
  Política de etiquetado para imágenes Docker.
\item
  Resultado de la verificación del archivo Docker Compose.
\item
  Documento explicando la implementación de la red de contenedores.
\item
  Archivo ``.dockerignore''.
\end{itemize}

\hypertarget{rubrica-de-evaluaciuxf3n-4}{%
\section{Rubrica de Evaluación:}\label{rubrica-de-evaluaciuxf3n-4}}

\begin{itemize}
\tightlist
\item
  Escaneo de vulnerabilidades realizado con éxito: 4 puntos
\item
  Correcta implementación de la política de etiquetado: 4 puntos
\item
  Validación exitosa del archivo Docker Compose: 4 puntos
\item
  Implementación adecuada de la red de contenedores: 4 puntos
\item
  Correcta configuración del archivo ``.dockerignore'': 4 puntos
\end{itemize}

Resolución de la Actividad Práctica

Escaneo de vulnerabilidades:

\begin{Shaded}
\begin{Highlighting}[]
\ExtensionTok{snyk}\NormalTok{ container test alpine:latest}
\end{Highlighting}
\end{Shaded}

Política de etiquetado (ejemplo):

Se etiquetarán las imágenes con el formato ``versión-año-mes-día''
(ejemplo: 1.0-20230115).

Verificación del archivo Docker Compose:

\begin{Shaded}
\begin{Highlighting}[]
\ExtensionTok{docker{-}compose}\NormalTok{ config}
\end{Highlighting}
\end{Shaded}

Implementación de una red de contenedores:

\begin{Shaded}
\begin{Highlighting}[]
\ExtensionTok{docker}\NormalTok{ network create my{-}network}
\end{Highlighting}
\end{Shaded}

Archivo ``.dockerignore'' (ejemplo):

\begin{Shaded}
\begin{Highlighting}[]
\ExtensionTok{node\_modules}
\ExtensionTok{.git}
\ExtensionTok{.env}
\end{Highlighting}
\end{Shaded}




\end{document}
